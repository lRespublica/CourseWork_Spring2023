\documentclass[a4paper, 14pt]{extarticle}
    
\usepackage{amsmath,amsthm,amssymb}
\usepackage{mathtext}
\usepackage{alltt}
\usepackage[T1,T2A]{fontenc}
\usepackage[utf8]{inputenc}
\usepackage[english,russian]{babel}
\usepackage{indentfirst}

\usepackage{setspace}

\usepackage{geometry}
\geometry{left=2cm}
\geometry{right=1.5cm}
\geometry{top=1cm}
\geometry{bottom=2cm}


\begin{document}
\begin{titlepage}
    \newpage
    \begin{center}
    {\bfseries 	Федеральное государственное автономное образовательное \\
				Учреждение высшего профессионального образования \\
				«Национальный исследовательский университет \\
				«Высшая школа экономики»}
    \vspace{1cm}
    
    {\bfseries  Факультет МИЭМ \\
				Департамент прикладной математики}
    
    \vspace{1cm}

    {\bfseries{\Large Курсовая работа}} \\
    По дисциплине <<Математический анализ>> \\
    для направления 01.03.04 <<Прикладная математика>>
    
    \vspace{1cm}
    
    {\bfseries{\large 	<<Элементарные асимптотические методы>>} \\
    					Вариант 41}
    
	\end{center}
	
	\vspace{\fill}
	
	\begin{flushright}
	{\bfseries	{\large Выполнил студент группы БПМ232} \\
						Знаменок Леонид}
	
	\vspace{1cm}					
	{\bfseries	{\large Руководитель курсовой работы} \\
			  			Белова Мария Владимировна}
	\end{flushright}
	
	\vspace{1cm}
    \begin{center}
    Москва 2023
    \end{center}

\end{titlepage}

{

\section{Задание 1}

\subsection{Условие}
Написать асимптотическую формулу для \(f(x) = (1 + ln\; cos\: x)^{\frac{1}{x^2}} \) при \(x \to 0\). В ответе должно быть не менее двух членов асимптотической формулы, не считая остатка. 

\subsection{Решение}
\doublespacing{
Преобразуем функцию: \[f(x) = (1 + ln\; cos\: x)^{\frac{1}{x^2}} = e^{\frac{1}{x^2} ln(1 + ln\; cos\: x)}\] 

Пусть \(ln\; cos\: x = \alpha(x);\quad \alpha(x) \to 0; \ x \to 0\) 

Тогда \(ln(1 + ln\; cos\: x) = ln(1 + \alpha(x)) = \alpha(x) - \frac{\alpha^2(x)}{2} + O(\alpha^3(x)); \quad \alpha(x) \to 0\)

\(\alpha(x) = ln\; cos\: x = ln (1 - \frac{x^2}{2} + \frac{x^4}{24} + O(x^6)) = ln(1 + \beta(x));\quad \beta(x) \to 0;\ x\to 0\)

\(\beta(x) = - \frac{x^2}{2} + \frac{x^4}{24} + O(x^6)\)

\(\beta^2(x) = \frac{x^4}{4} - \frac{x^6}{24} + O(x^8)\)

\(\alpha(x) = \beta(x) - \frac{\beta^2(x)}{2} + O(\beta^3(x)) = - \frac{x^2}{2} + \frac{x^4}{24} + O(x^6) - \frac{x^4}{8} + \frac{x^6}{48} + O(x^8) + O(x^6) =\)

\(= - \frac{x^2}{2} - \frac{x^4}{12} + O(x^6)\)

\(\alpha^2(x) = \frac{x^4}{4} + \frac{x^6}{12} + O(x^8)\)

\(ln(1 + \alpha(x)) = \alpha(x) - \frac{\alpha^2(x)}{2} + O(\alpha^3(x)) = -\frac{x^2}{2} - \frac{x^4}{12} + O(x^6) - \frac{x^4}{4} - \frac{x^6}{12} + O(x^8) + O(x^6) = -\frac{x^2}{2} - \frac{5x^4}{24} + O(x^6)\)

\vspace{0.5cm}
\(f(x) = e ^ {\frac{1}{x^2}(-\frac{x^2}{2} - \frac{5x^4}{24} + O(x^6))} = e^{- \frac{1}{2} - \frac{5x^2}{24} + O(x^4)} = \sqrt{e}^{-1} * e^{\gamma(x)};\quad \gamma(x) \to 0;\  x \to 0 \)

\(e^{\gamma(x)} = 1 + \gamma(x) + O(\gamma^2(x)) = 1 - \frac{5x^2}{24} + O(x^4)\)

\[f(x) = \frac{1}{\sqrt{e}}(1-\frac{5x^2}{24}+O(x^4))\]
}

\newpage

\section{Задание 2}

\subsection{Условие}
Написать асимптотическую формулу для \(f(x) = \sqrt{x^2+1} - \sqrt[3]{\frac{x^4}{x-1}} \) при \(x \to +\infty\). В ответе должно быть не менее двух членов асимптотической формулы, не считая остатка. 

\subsection{Решение}
\doublespacing{
\(f(x) = \sqrt{x^2+1} - \sqrt[3]{\frac{x^4}{x-1}}  = x\sqrt{1+\frac{1}{x^2}} - x\sqrt[3]{\frac{x}{x-1}} = x\sqrt{1+\frac{1}{x^2}} - x\sqrt[3]{(1 - \frac{1}{x})^{-1}} = x((1+\frac{1}{x^2})^{\frac{1}{2}} - (1 -\frac{1}{x})^{-\frac{1}{3}}) = x(\alpha(x) - \beta(x));\quad x \to 0\)

\(\alpha(x) = (1+\frac{1}{x^2})^{\frac{1}{2}} = 1 + \frac{1}{2x^2} + O(\frac{1}{x^4}) \)

\(\beta(x) = (1 -\frac{1}{x})^{-\frac{1}{3}} = 1 + \frac{1}{3x} + \frac{2}{9x^2} + O(\frac{1}{x^3})\)

\(\alpha(x) - \beta(x) = -\frac{1}{3x} + \frac{5}{18x^2} + O(\frac{1}{x^3})\)

\(f(x) = x(-\frac{1}{3x} + \frac{5}{18x^2} + O(\frac{1}{x^3})) = -\frac{1}{3} + \frac{5}{18x} + O(\frac{1}{x^2})\)

}
\end{document}