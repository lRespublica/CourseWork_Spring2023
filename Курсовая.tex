\documentclass[a4paper, 14pt]{extarticle}
    
\usepackage{amsmath,amsthm,amssymb}
\usepackage{mathtext}
\usepackage{alltt}
\usepackage[T1,T2A]{fontenc}
\usepackage[utf8]{inputenc}
\usepackage[english,russian]{babel}
\usepackage{indentfirst}

\usepackage{setspace}

\usepackage{geometry}
\geometry{left=2cm}
\geometry{right=1.5cm}
\geometry{top=1cm}
\geometry{bottom=2cm}


\begin{document}
\begin{titlepage}
    \newpage
    \begin{center}
    {\bfseries 	Федеральное государственное автономное образовательное \\
				Учреждение высшего профессионального образования \\
				«Национальный исследовательский университет \\
				«Высшая школа экономики»}
    \vspace{1cm}
    
    {\bfseries  Факультет МИЭМ \\
				Департамент прикладной математики}
    
    \vspace{1cm}

    {\bfseries{\Large Курсовая работа}} \\
    По дисциплине <<Математический анализ>> \\
    для направления 01.03.04 <<Прикладная математика>>
    
    \vspace{1cm}
    
    {\bfseries{\large 	<<Элементарные асимптотические методы>>} \\
    					Вариант 41}
    
	\end{center}
	
	\vspace{\fill}
	
	\begin{flushright}
	{\bfseries	{\large Выполнил студент группы БПМ232} \\
						Знаменок Леонид}
	
	\vspace{1cm}					
	{\bfseries	{\large Руководитель курсовой работы} \\
			  			Белова Мария Владимировна}
	\end{flushright}
	
	\vspace{1cm}
    \begin{center}
    Москва 2023
    \end{center}

\end{titlepage}



\section{Задание 1}

\subsection{Условие}
Написать асимптотическую формулу для \(f(x) = (1 + ln\; cos\: x)^{\frac{1}{x^2}} \) при \(x \to 0\). В ответе должно быть не менее двух членов асимптотической формулы, не считая остатка. 

\subsection{Решение}
\doublespacing{
Преобразуем функцию: \[f(x) = (1 + ln\; cos\: x)^{\frac{1}{x^2}} = e^{\frac{1}{x^2} ln(1 + ln\; cos\: x)}\] 

Пусть \(ln\; cos\: x = \alpha(x);\quad \alpha(x) \to 0; \ x \to 0\) 

Тогда \(ln(1 + ln\; cos\: x) = ln(1 + \alpha(x)) = \alpha(x) - \frac{\alpha^2(x)}{2} + O(\alpha^3(x)); \quad \alpha(x) \to 0\)

\(\alpha(x) = ln\; cos\: x = ln (1 - \frac{x^2}{2} + \frac{x^4}{24} + O(x^6)) = ln(1 + \beta(x));\quad \beta(x) \to 0;\ x\to 0\)

\(\beta(x) = - \frac{x^2}{2} + \frac{x^4}{24} + O(x^6)\)

\(\beta^2(x) = \frac{x^4}{4} - \frac{x^6}{24} + O(x^8)\)

\(\alpha(x) = \beta(x) - \frac{\beta^2(x)}{2} + O(\beta^3(x)) = - \frac{x^2}{2} + \frac{x^4}{24} + O(x^6) - \frac{x^4}{8} + \frac{x^6}{48} + O(x^8) + O(x^6) =\)

\(= - \frac{x^2}{2} - \frac{x^4}{12} + O(x^6)\)

\(\alpha^2(x) = \frac{x^4}{4} + \frac{x^6}{12} + O(x^8)\)

\(ln(1 + \alpha(x)) = \alpha(x) - \frac{\alpha^2(x)}{2} + O(\alpha^3(x)) = -\frac{x^2}{2} - \frac{x^4}{12} + O(x^6) - \frac{x^4}{8} - \frac{x^6}{24} + O(x^8) + O(x^6) = -\frac{x^2}{2} - \frac{5x^4}{24} + O(x^6)\)

\vspace{0.5cm}
\(f(x) = e ^ {\frac{1}{x^2}(-\frac{x^2}{2} - \frac{5x^4}{24} + O(x^6))} = e^{- \frac{1}{2} - \frac{5x^2}{24} + O(x^4)} = \sqrt{e}^{-1} * e^{\gamma(x)};\quad \gamma(x) \to 0;\  x \to 0 \)

\(e^{\gamma(x)} = 1 + \gamma(x) + O(\gamma^2(x)) = 1 - \frac{5x^2}{24} + O(x^4)\)

\[f(x) = \frac{1}{\sqrt{e}}(1-\frac{5x^2}{24}+O(x^4))\]
}

\newpage

\section{Задание 2}

\subsection{Условие}
Написать асимптотическую формулу для \(f(x) = \sqrt{x^2+1} - \sqrt[3]{\frac{x^4}{x-1}} \) при \(x \to +\infty\). В ответе должно быть не менее двух членов асимптотической формулы, не считая остатка. 

\subsection{Решение}
\doublespacing{
\(f(x) = \sqrt{x^2+1} - \sqrt[3]{\frac{x^4}{x-1}}  = x\sqrt{1+\frac{1}{x^2}} - x\sqrt[3]{\frac{x}{x-1}} = x\sqrt{1+\frac{1}{x^2}} - x\sqrt[3]{(1 - \frac{1}{x})^{-1}} = x((1+\frac{1}{x^2})^{\frac{1}{2}} - (1 -\frac{1}{x})^{-\frac{1}{3}}) = x(\alpha(x) - \beta(x));\quad x \to +\infty\)

\(\alpha(x) = (1+\frac{1}{x^2})^{\frac{1}{2}} = 1 + \frac{1}{2x^2} + O(\frac{1}{x^4});\quad \frac{1}{x} \to 0; \)

\(\beta(x) = (1 -\frac{1}{x})^{-\frac{1}{3}} = 1 + \frac{1}{3x} + \frac{2}{9x^2} + O(\frac{1}{x^3});\quad \frac{1}{x} \to 0;\)

\(\alpha(x) - \beta(x) = -\frac{1}{3x} + \frac{5}{18x^2} + O(\frac{1}{x^3})\)

\(f(x) = x(-\frac{1}{3x} + \frac{5}{18x^2} + O(\frac{1}{x^3})) = -\frac{1}{3} + \frac{5}{18x} + O(\frac{1}{x^2})\)
}


\newpage

\section{Задание 3}
\subsection{Условие}
Используя формулу Тейлора, найти асимптотику корней уравнения \[sin\: x \ - \frac{1}{x^2} = 0, x> 0\] В ответе должно быть не менее двух членов асимптотики.
\subsection{Решение}
\doublespacing {
Рассмотрим уравнение \(sin\: x = \frac{1}{x^2}\).

При \(x \to +\infty\quad \frac{1}{x^2} \to 0\)

Соответственно, последовательность корней уравнений представима в виде \(x_n = \pi n + \alpha_n;\quad \alpha_n \to 0;\ n \to +\infty \)

Подставим \(x_n\) в обе части уравнения:
\begin{enumerate}
\item \(sin(\pi n +\alpha_n) = sin(\alpha_n)cos(\pi n) \; + \; cos(\alpha_n)sin(\pi n) = sin(\alpha_n)cos(\pi n) \sim \alpha_n cos(\pi n) = (-1)^n \alpha_n; \quad \alpha_n \to 0;\ n \to +\infty\)

\item \(\frac{1}{(\pi n + \alpha_n)^2} \sim \frac{1}{\pi^2 n^2}; \quad \alpha_n \to 0;\ n \to +\infty\)
\end{enumerate}

\((-1)^n \alpha_n = \frac{1}{\pi^2 n^2} + o(\frac{1}{n^2})\)

\(\alpha_n = (-1)^n \frac{1}{\pi^2 n^2} + o(\frac{1}{n^2})\)

Следовательно, \[x_n = \pi n + (-1)^n \frac{1}{\pi^2 n^2} + o(\frac{1}{n^2})\]
}

\section{Задание 4}
\subsection{Условие}
Написать асимптотическое представление функции \[F(x) = \int_x^{+\infty} \frac{cos(\sqrt{t+1})}{t\sqrt{t+1}} dt,\quad x \to +\infty\] содержащее два члена асимптотической последовательности.

\subsection{Решение}
\doublespacing {
Воспользуемся формулой интегрирования по частям:

\(\int_x^{+\infty} \frac{cos(\sqrt{t+1})}{t\sqrt{t+1}} dt = 
\small{
\begin{bmatrix} 
u = \frac{1}{t}; & du = -\frac{1}{t^2}dt\\
dv = \frac{cos(\sqrt{t+1})}{\sqrt{t+1}} dt & v = 2sin\sqrt{t+1}
\end{bmatrix}} 
= \left. \frac{2sin\sqrt{t+1}}{t}\right|_x^{+\infty} + 2\int_x^{+\infty} 
\frac{sin(\sqrt{t+1})}{t^2} dt = -\frac{2sin\sqrt{x+1}}{x} + 2\int_x^{+\infty} 
\frac{sin(\sqrt{t+1})}{t^2} dt\) 

\hspace{1cm}

Рассмотрим отдельно

\(\int_x^{+\infty} \frac{sin(\sqrt{t+1})}{t^2} dt = \small{
\begin{bmatrix}
\sqrt{t+1} = r; & r^2 = t +1 \\
t = r^2 - 1 & dt = 2rdr \\
\end{bmatrix}
\begin{bmatrix}
t & r\\
x & \sqrt{x+1}\\
+\infty & +\infty
\end{bmatrix}} = 2\int_{_{\sqrt{x+1}}}^{_{+\infty}}\frac{r\; sin\:r}{(r^2-1)^2}dr = 
\small{
\begin{bmatrix} 
u = \frac{r}{(r^2-1)^2}; & du = \frac{(r^2-1)^2 - 4r^2(r^2-1)}{(r^2-1)^4}dr\\
dv = sin\:r\ dr & v = -cos\:r
\end{bmatrix}} 
= \left. \frac{-2\ r \cos\:r\ }{(r^2-1)^2} \right|_{\sqrt{x+1}}^{+\infty} + \int_{_{\sqrt{x+1}}}^{_{+\infty}} \frac{(r^2-1)^2 - 4r^2(r^2-1)}{(r^2-1)^4}\cos r dr = \frac{2\sqrt{x+1}\cos\sqrt{x+1}}{x^2} + o(F(x))\) 

\hspace{1cm}

\(F(x) = -\frac{2sin\sqrt{x+1}}{x} + \frac{4\sqrt{x+1} \cos\sqrt{x+1}}{x^2} + o(\frac{1}{x^{3/2}}),\quad x \to +\infty\)
}

\newpage

\section{Задание 5}
\subsection{Условие}
Написать асимптотическое представление функции \(F(x) = \int_1^x \sqrt{t^2+1}e^{1/t}dt\) доведённое до члена, являющимся бесконечно малой функцией, при \(x \to +\infty\).

\subsection{Решение}
\doublespacing {
Рассмотрим отдельно
\begin{enumerate}
\item \(\sqrt{t^2 + 1} = t(1+\frac{1}{t^2})^{1/2} = t(1+\frac{1}{2t^2} + O(\frac{1}{t^4})) = t + \frac{1}{2t} + O(\frac{1}{t^3});\quad t \to +\infty\)

\item \(e^{1/t} = 1 + \frac{1}{t} + \frac{1}{2t^2} + O(\frac{1}{t^3});\quad t \to +\infty\)


\item \(\sqrt{t^2 + 1} *e^{1/t} = t + 1 + \frac{1}{2t} + O(\frac{1}{t^2}) + \frac{1}{2t} + \frac{1}{2t^2} + \frac{1}{4t^3} + O(\frac{1}{t^4}) + O(\frac{1}{t^3}) + O(\frac{1}{t^4}) + O(\frac{1}{t^5}) + O(\frac{1}{t^6}) = t + 1 + \frac{1}{t} + O(\frac{1}{t^2})\)
\end{enumerate}

Вернёмся к изначальному интегралу:

\(F(x) = \int_1^x \sqrt{t^2+1}e^{1/t}dt = \int_1^x (\sqrt{t^2+1}e^{1/t} - t - 1 - \frac{1}{t})dt\ +\ \int_1^x (t + 1 + \frac{1}{t})dt = F_1(x) + \left.(\frac{t^2}{2} + t +ln\: t)\right|_1^x = F_1(x) + \frac{x^2}{2} + x + ln\: x - \frac{3}{2}\)

Осталось исследовать асимптотику функции \[F_1(x) = \int_1^x (\sqrt{t^2+1}e^{1/t} - t - 1 - \frac{1}{t})dt\]

Так как в последнем выражении подынтегральная функция имеет асимптотику \(O(\frac{1}{t^2}),\  t \to + \infty\), то несобственный интеграл \(F_1(x)\) сходится. Обозначим его значение за A. Тогда

\(F_1(x) = A - \int_x^{+\infty} (\sqrt{t^2+1}e^{1/t} - t - 1 - \frac{1}{t})dt = A - \int_x^{+\infty}O(\frac{1}{t^2})dt  = A + O(\frac{1}{x})\) 
}

Итак,
\[F(x) = \frac{x^2}{2} + x + ln\: x + (A-\frac{3}{2}) + O(\frac{1}{x}), \quad x \to + \infty\]

\end{document}